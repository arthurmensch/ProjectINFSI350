\documentclass[10pt,a4paper]{article}
\usepackage[utf8]{inputenc}
\usepackage[T1]{fontenc}
\usepackage{amsmath}
\usepackage{amsfonts}
\usepackage{amssymb}
\usepackage{graphicx}
\graphicspath{ {./figures/} }
\usepackage{lmodern}
\usepackage{array}
\usepackage[a4paper, margin=20mm]{geometry}
\usepackage[frenchb]{babel}
\usepackage{subcaption}


\author{Arthur Mensch, Paul Vallet, Michaël Weiss}
\title{Déformation de cage par coordonnées de Green}
\begin{document}
\maketitle
\begin{abstract}
Nous présentons dans ce rapport notre implémentation de l'article \cite{lipman2008green}, qui propose une méthode pour déformer un maillage à partir de la déformation d'une cage de contrôle, en utilisant les coordonnées de Green. Nous revenons sur les bases théoriques de la déformation et présentons les fonctionnalités et détails d'implémentations de notre programme.
\end{abstract}
\section*{Introduction}



\section{Principe}

\subsection{Déformation par cage}

\subsection{Coordonnées de Green}

\section{Démonstration}

\subsection{Fonctionnalités}

\subsection{Fonctionnement}

\section*{Conclusion}

\section*{Référence}

\bibliographystyle{plain}

\bibliography{report}

\end{document}